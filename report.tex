\documentclass[12pt]{article}
\usepackage[a4paper, margin=1in]{geometry}
\usepackage{setspace}
\usepackage{times}

\setstretch{1.5}

\begin{document}

\section{Introduction}

In many universities, students are still required to visit banks physically to pay their semester fees. This traditional method is time-consuming, inconvenient, and often involves long queues and manual paperwork. As a result, students waste valuable academic time and face unnecessary stress.

Additionally, most official university notices and updates are shared through Facebook groups. Many students miss important information because they do not regularly check social media, or important posts get buried among other content. This creates communication gaps between the university and students.

To solve these problems, an online system is proposed that allows students to pay fees digitally and receive all official notifications in one platform.

\section{Overview of the System}

The proposed system is a web-based and mobile-friendly platform designed for university students and administrators. It will provide a centralized system for fee payment, academic document access, and official communication.

\subsection{Student Features}

Students will be able to:

\begin{itemize}
    \item Pay semester fees using digital payment services such as bKash and Nagad
    \item Receive automatic digital payment receipts
    \item Download admit cards online
    \item View official university notices and announcements
\end{itemize}

\subsection{Administrator Features}

University administrators will be able to:

\begin{itemize}
    \item Manage student payment records
    \item Verify transactions
    \item Upload notices and announcements
    \item Generate reports
\end{itemize}

The system will ensure secure transactions, accurate record-keeping, and easy access to important information.
\section{Problem Statement and Analysis}

Currently, students must visit banks physically to pay semester fees, which causes long waiting times, transportation costs, and unnecessary stress. Manual payment systems are prone to errors, delays in verification, and loss of payment records. Students often face difficulties in collecting receipts and confirming successful payments.

Moreover, important university notices are mainly shared through Facebook groups. Many students miss these updates due to irregular social media usage, inactive accounts, or posts being buried under other content. This leads to misinformation, missed deadlines, and poor communication between students and the administration.

The absence of a centralized digital platform results in inefficiency, lack of transparency, and dependency on manual processes. Therefore, there is a strong need for an integrated online system that can manage fee payments and official notifications in a secure and organized manner.

\section{Objectives and Scope}

\subsection{Objectives}

The main objectives of this system are:

\begin{itemize}
    \item To reduce the need for physical bank visits
    \item To save time and effort for students and staff
    \item To provide a secure and reliable online payment system
    \item To ensure that students receive all important notices on time
    \item To digitize administrative processes
\end{itemize}

\subsection{Scope}

The scope of this system includes:

\begin{itemize}
    \item Online fee payment using bKash and Nagad
    \item Automatic generation of digital receipts
    \item Admit card download facility
    \item Centralized notice board
    \item User authentication for students and administrators
    \item Payment and notice management system
\end{itemize}

The system does not cover academic grading, course registration, or learning management features in its initial version.

\section{Problem Statement and Analysis}

Currently, students must visit banks physically to pay semester fees, which causes long waiting times, transportation costs, and unnecessary stress. Manual payment systems are prone to errors, delays in verification, and loss of payment records. Students often face difficulties in collecting receipts and confirming successful payments.

Moreover, important university notices are mainly shared through Facebook groups. Many students miss these updates due to irregular social media usage, inactive accounts, or posts being buried under other content. This leads to misinformation, missed deadlines, and poor communication between students and the administration.

The absence of a centralized digital platform results in inefficiency, lack of transparency, and dependency on manual processes. Therefore, there is a strong need for an integrated online system that can manage fee payments and official notifications in a secure and organized manner.

\section{Requirement Analysis}

Requirement analysis defines the services and constraints of the proposed system. It helps in understanding what the system should do and how it should perform.

\subsection{Functional Requirements}

The system must be able to:

\begin{itemize}
    \item Allow students to register and log in securely
    \item Enable students to pay semester fees using bKash and Nagad
    \item Generate and store digital payment receipts
    \item Allow students to download admit cards
    \item Display all official notices and announcements
    \item Allow administrators to verify payments
    \item Enable administrators to upload, update, and delete notices
    \item Generate payment and activity reports
    \item Maintain student and transaction records
\end{itemize}

\subsection{Non-Functional Requirements}

The system should ensure:

\begin{itemize}
    \item \textbf{Security:} Protection of user data and secure payment processing
    \item \textbf{Performance:} Fast response time and smooth transaction handling
    \item \textbf{Reliability:} Continuous availability with minimum downtime
    \item \textbf{Usability:} User-friendly interface for students and staff
    \item \textbf{Scalability:} Ability to handle increasing number of users
    \item \textbf{Maintainability:} Easy system updates and error fixing
\end{itemize}
\end{document}
