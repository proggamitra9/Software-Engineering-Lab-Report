\documentclass[12pt]{article}
\usepackage[a4paper, margin=1in]{geometry}
\usepackage{setspace}
\usepackage{times}

\setstretch{1.5}

\begin{document}

\section{Introduction}

In many universities, students are still required to visit banks physically to pay their semester fees. This traditional method is time-consuming, inconvenient, and often involves long queues and manual paperwork. As a result, students waste valuable academic time and face unnecessary stress.

Additionally, most official university notices and updates are shared through Facebook groups. Many students miss important information because they do not regularly check social media, or important posts get buried among other content. This creates communication gaps between the university and students.

To solve these problems, an online system is proposed that allows students to pay fees digitally and receive all official notifications in one platform.

\section{Overview of the System}

The proposed system is a web-based and mobile-friendly platform designed for university students and administrators. It will provide a centralized system for fee payment, academic document access, and official communication.

\subsection{Student Features}

Students will be able to:

\begin{itemize}
    \item Pay semester fees using digital payment services such as bKash and Nagad
    \item Receive automatic digital payment receipts
    \item Download admit cards online
    \item View official university notices and announcements
\end{itemize}

\subsection{Administrator Features}

University administrators will be able to:

\begin{itemize}
    \item Manage student payment records
    \item Verify transactions
    \item Upload notices and announcements
    \item Generate reports
\end{itemize}

The system will ensure secure transactions, accurate record-keeping, and easy access to important information.
\section{Problem Statement and Analysis}

Currently, students must visit banks physically to pay semester fees, which causes long waiting times, transportation costs, and unnecessary stress. Manual payment systems are prone to errors, delays in verification, and loss of payment records. Students often face difficulties in collecting receipts and confirming successful payments.

Moreover, important university notices are mainly shared through Facebook groups. Many students miss these updates due to irregular social media usage, inactive accounts, or posts being buried under other content. This leads to misinformation, missed deadlines, and poor communication between students and the administration.

The absence of a centralized digital platform results in inefficiency, lack of transparency, and dependency on manual processes. Therefore, there is a strong need for an integrated online system that can manage fee payments and official notifications in a secure and organized manner.

\end{document}
